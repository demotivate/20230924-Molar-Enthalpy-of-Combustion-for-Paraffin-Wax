%%%%%%%%%%%%%%%%%%%%%%%%%%%%%%%%%%%%%%%%%
% University/School Laboratory Report
% LaTeX Template
% Version 4.0 (March 21, 2022)
%
% This template originates from:
% https://www.LaTeXTemplates.com
%
% Authors:
% Vel (vel@latextemplates.com)
% Linux and Unix Users Group at Virginia Tech Wiki
%
% License:
% CC BY-NC-SA 4.0 (https://creativecommons.org/licenses/by-nc-sa/4.0/)
%
% Modifications made:
% 1. Replaced title, name, date, date performed, and instructor with personal details
% 2. Removed partners row
% 3. Replaced original structure with personal structure
% 
%%%%%%%%%%%%%%%%%%%%%%%%%%%%%%%%%%%%%%%%%

%----------------------------------------------------------------------------------------
%	PACKAGES AND DOCUMENT CONFIGURATIONS
%----------------------------------------------------------------------------------------

\documentclass[
	letterpaper, % Paper size, specify a4paper (A4) or letterpaper (US letter)
	12pt, % Default font size, specify 10pt, 11pt or 12pt
]{CSUniSchoolLabReport}

\nocite{*}
\addbibresource{bibliography.bib} % Bibliography file (located in the same folder as the template)

%----------------------------------------------------------------------------------------
%	REPORT INFORMATION
%----------------------------------------------------------------------------------------

\title{Determination of the Molar Enthalpy \\ of Combustion of Paraffin Wax \\ IB Chemistry HL} % Report title

\author{Ethan \textsc{Chen}} % Author name(s), add additional authors like: '\& James \textsc{Smith}'

\date{September 29, 2023} % Date of the report

%----------------------------------------------------------------------------------------

\begin{document}

\maketitle % Insert the title, author and date using the information specified above

\begin{center}
  \begin{tabular}{l r}
    Date Performed: & September 18, 2023 \\ % Date the experiment was performed
    Instructor:     & Mr \textsc{Deis}   % Instructor/supervisor
  \end{tabular}
\end{center}

% If you need to include an abstract, uncomment the lines below
%\begin{abstract}
%	Abstract text
%\end{abstract}

%----------------------------------------------------------------------------------------
%	OBJECTIVE
%----------------------------------------------------------------------------------------

\section{Purpose}

\par The purpose of this lab was to determine the molar enthalpy of the combustion of Paraffin Wax (\ce{C25H52(s)}) in the following reaction:

\begin{center}
  \ce{C25H52(s) + 38O2(g) -> 25CO2(g) + 26H2O(g)}
\end{center}


\section{Evidence}

\subsection{Qualitative Observations}

% Paraffin Wax
The Paraffin Wax candle appears white and opaque, with the colour fading into orange nearing the bottom.
Upon touch, the Paraffin Wax feels slightly slippery, and after putting the candle down, the slippery sensation
transfers to your hands.
\\
\par The Paraffin Wax candle was able to be attached to the watch glass by exposing the
bottom of the candle to a flame, in which the melted paraffin wax at the bottom
was able to stick onto the watch glass. This indicates that melted paraffin wax exhibits
adhesive properties.
\\
\par The container surrounding the base of the Paraffin Wax candle does not leave
any openings for air to escape other than the opening at the top.
\\
\par These observations are shown in Figure \ref*{fig:paraffin}.

\begin{figure}[H] % [H] forces the figure to be placed exactly where it appears in the text
  \centering % Horizontally center the figure
  \includegraphics[width=0.65\textwidth]{Paraffin Wax} % Include the figure
  \caption{Appearance of the Paraffin Wax candle attached to watch glass within container.}
  \label{fig:paraffin}
\end{figure}

% Can holding water Top View
\par The can containing the water has a proportionally greater height than diameter.
The opening at the top of the can is significantly wider than the diameter of the thermometer,
indicating that the space within the can is not entirely closed and isolated.
The two holes allowing for the rod to pass through do wrap around the rod
tightly, however because it is a manually created hole, it still allows for passage
of air within the can and outside the can. These observations are shown in
Figure \ref*{fig:cantop}.
\begin{figure}[H]
  \centering
  \includegraphics[width=0.65\textwidth]{top}
  \caption{Appearance of can used to contain the water from a top view}
  \label{fig:cantop}
\end{figure}

%Aluminium Foil
\par The Aluminium Foil surrounding the can leaves an opening at the top between
the ring and the diameter of the can, as seen in Figure \ref*{fig:cantop}.
\\
\par The process of conducting a trial involved raising the ring along the Ring Stand,
lighting the candle, and lowering the ring. This means that there is an interval of time
from lighting the candle where combustion was not taking place within the calorimeter.
Additionally, after lowering the ring to a reasonable height where the bottom of the
aluminium foil surrounding the can meets the top of the container surrounding the
candle, the two ends of the aluminium foil become separated due to the significantly
larger diameter of the container at the bottom. This leaves a few openings at the side
of the aluminium foil cylinder.
\\
\par Some of these observations can be seen in Figure \ref*{fig:canside}.
\begin{figure}[H]
  \centering
  \includegraphics[width=0.65\textwidth]{side}
  \caption{View of aluminium foil attached around elevated ring on ring stand}
  \label{fig:canside}
\end{figure}

%Can bottom
\par As seen in Figure \ref*{fig:canbottom}, there is a black stain on the bottom of the can
containing the bottom. This is presumably solid carbon produced by incomplete combustion
from previous calorimetry experiments.
\begin{figure}[H]
  \centering
  \includegraphics[width=0.65\textwidth]{bottom}
  \caption{Appearance of black stain on the bottom of the can containing water}
  \label{fig:canbottom}
\end{figure}

\subsection{Quantitative Data}

\par The raw data recorded from the experiment is recorded in Table \ref*{tab:rawdata}.
Note that when measuring the mass of the water, the lab scale was tared
with the can, then water was added into the can. Therefore, the mass recorded
doesn't need to be subtracted by the mass of the can.

\begin{table}[H]
  \centering
  \resizebox{\linewidth}{!}{%
    \begin{tblr}{
      width = \linewidth,
      colspec = {|Q[406]|Q[169]|Q[169]|Q[169]|},
      }
      \hline
      Trial Number & 1 & 2 & 3 \\
      \hline
      {Mass of can /g          \\(±0.003g)} & 13.039 & 15.769 & 15.080\\
      \hline
      {Mass of water /g        \\(±0.003g)} & 50.141 & 50.112 & 50.105\\
      \hline
      {Initial mass of candle  \\and watch glass /g\\(±0.003g)} & 106.907 & 106.209 & 105.691\\
      \hline
      {Final mass of candle    \\and watch glass /g\\(±0.003g)} & 106.209 & 105.691 & 105.240\\
      \hline
      {Initial Temperature     \\of water /\textcolor[rgb]{0.302,0.318,0.337}{°}C\\(±0.3~\textcolor[rgb]{0.302,0.318,0.337}{°}C)} & 23.8 & 25.0 & 25.0\\
      \hline
      {Final Temperature       \\of water /\textcolor[rgb]{0.302,0.318,0.337}{°}C\\(±0.3~\textcolor[rgb]{0.302,0.318,0.337}{°C})} & 60.7 & 51.2 & 52.6\\
      \hline
    \end{tblr}
  }
  \caption{Raw data recorded from each trial during the experiment}
  \label{tab:rawdata}
\end{table}


\par Something else to keep in mind is that the final temperature of the water was determined by
setting a timer for 5 minutes and recording the temperature afterwards. However,
for the first trial, we forgot to consider after what period of time would we record
the final temperature. Therefore, that trial simply involved noting the temperature after twenty
15 second intervals which was manually counted. This demonstrates a reason why
the final temperature of the water for the first trial was significantly different than
the two other trials.

\section{Analysis}

\subsection{Calculation of theoretical molar enthalpy\\ of combustion}

\par The theoretical molar enthalpy of combustion for Paraffin Wax \\(\ce{C25H52(s)}) will be
calculated using Hess' Law.

\par With \(\Delta H\degree_{f}\) being the molar enthalpy of formation for some compound,
the following values will be used.
\\
\\
\begin{tabular}{ll}
  \centering
  \ce{C25H52(s)}: & \(\Delta H\degree_{f} = -1424.3~kJmol^{-1}\) \\
  \ce{CO2(g)}:    & \(\Delta H\degree_{f} = -393.5~kJmol^{-1}\)  \\
  \ce{H2O(g)}:    & \(\Delta H\degree_{f} = -241.8~kJmol^{-1}\)
\end{tabular}
\\
\\

\par These values can then be used to calculated to theoretical molar enthalpy of combustion for Paraffin Wax with reference to the balanced chemical equation of the combustion of Paraffin Wax.
\\
\ce{C25H52(s) + 38O2(g) -> 25CO2(g) + 26H2O(g)}

\begin{align*}
  \text{let}~ & \Delta H\degree_{comb} = \mbox{calculated theoretical molar enthalpy of combustion}             \\
              & \text{ for Paraffin Wax } /kJ                                                                   \\
              & n                          = \mbox{the amount of a compound } /mol                              \\
              & \Delta H\degree_{f}        = \mbox{the molar enthalpy of formation for a compound } /kJmol^{-1}
\end{align*}
\begin{align*}
  \Delta H\degree_{comb} & = \sum nH\degree_{f}products - \sum nH\degree_{f}reactants
  \\
  \Delta H\degree_{comb} & = \left((25mol)(-393.5kJmol^{-1}) + (26mol)(-241.8kJmol^{-1})\right)
  \\ & - (1mol)(-1424.3kJmol^{-1})
  \\
  \Delta H\degree_{comb} & = -14700kJ\
  \\
  \Delta H\degree_{comb} & = -14.7MJ\
\end{align*}

\subsection{Calculation of experimental molar enthalpy\\ of combustion}
\par The system of the calorimetry experiment will be the combustion of the Paraffin Wax,
and the surroundings will be the water and the can. We will make the assumption that
the can is made out of aluminium (\cite{Drink_can_2023}).

\par The systems and surroundings are presented in Table \ref*{tab:system} along with
what they entail. Note that \(q\) equals to the total heat in J.

\begin{table}[H]
  \centering
  \resizebox{\linewidth}{!}{%
    \begin{tabular}{>{\hspace{0pt}}m{0.369\linewidth}|>{\hspace{0pt}}m{0.569\linewidth}}
      System                                                                       & Surroundings                                                                                                                     \\
      \hline
      Combustion of Paraffin Wax\par{}\(\Delta E_p\)\par{}\(q_1=nH\degree_{comb}\) & Water + Aluminium Can\par{}\(\Delta E_k\)\par{}\(q_2=m_{water}c_{water}\Delta T_{water}\)\par{}\(q_3=m_{Al}c_{Al}\Delta T_{Al}\)
    \end{tabular}
  }
  \caption{System and surroundings of the calorimeter.}
  \label{tab:system}
\end{table}

\par Given that \(q_1=q_2+q_3\), the value of \(nH\degree_{comb}\) can then be calculated. A sample calculation is shown below for Trial 1.
\begin{align*}
   & q_1              = q_2 + q_3                                                      \\
   & nH\degree_{comb} = m_{water}c_{water}\Delta T_{water} + m_{Al}c_{Al}\Delta T_{Al}
\end{align*}
\begin{align*}
  \text{where } & H\degree_{comb} = \text{ calculated experimental molar enthalpy of combustion }
  \\ & \text{of Paraffin Wax } /J
  \\ & n = \text{ amount of Paraffin Wax combusted } /mol
  \\ & m = \text{ mass of a component in the surroundings } /g
  \\ & c = \text{ specific heat capacity of a component in the } \\ & \text{surroundings } /Jg^{-1}\degree C^{-1}
  \\ & \Delta T = \text{ change in temperature of a component in the } \\ & \text{surroundings } /\degree C
  \\ H\degree_{comb} & = \frac{m_{water}c_{water}\Delta T_{water} + m_{Al}c_{Al}\Delta T_{Al}}{n}
\end{align*}
In order to be able to calculate $n$, the molar mass of Paraffin Wax (\ce{C25H52(s)}) must be calculated first.
\begin{align*}
                & M_{\ce{C25H52(s)}} = \sum (M_{element} \times n_{element})  \\
  \text{where } & M = \text{ molar mass of an element or compound } gmol^{-1}
  \\ & n = \text{ amount of an element or compound } /mol
  \\ & M_{\ce{C25H52(s)}} = (25 mol)(12.02 gmol^{-1}) + (52 mol)(1.01 gmol^{-1})                                       \\
                & = 352.77 g
\end{align*}
\\
\begin{align*}
   & m_{water}c_{water}\Delta T_{water} + m_{Al}c_{Al}\Delta T_{Al}
  \\ & = (50.141g \pm 0.003g)(4.19Jg^{-1}\degree C^{-1})(60.7\degree C \pm 0.3\degree C - 23.8\degree C \pm 0.3\degree C)
  \\ & + (13.039g \pm 0.003g)(0.897Jg^{-1}\degree C^{-1})(60.7\degree C \pm 0.3\degree C - 23.8\degree C \pm 0.3\degree C)
  \\
  \\
   & m_{water}c_{water}\Delta T_{water} + m_{Al}c_{Al}\Delta T_{Al}
  \\ & = (50.141g \pm 0.006\%)(4.19Jg^{-1}\degree C^{-1})(36.9\degree C \pm 2\%)
  \\ & + (13.039g \pm 0.02\%)(0.897Jg^{-1}\degree C^{-1})(36.9\degree C \pm 2\%)
  \\
  \\
   & m_{water}c_{water}\Delta T_{water} + m_{Al}c_{Al}\Delta T_{Al} = 8.18 \times 10^3J \pm 4\%
  \\
  \\ & n                = (106.907 \pm 0.003g - 106.209g \pm 0.003g)(\frac{1mol}{352.77g})
  \\
   & n                = (0.698g \pm 0.9\%)(\frac{1mol}{352.77g})
  \\
   & n                = 1.98 \times 10^{-3} mol \pm 0.9\%
  \\
  \\
   & H\degree_{comb}  = \frac{8.18 \times 10^3J \pm 4\%}{1.98 \times 10^{-3} mol \pm 0.9\%}
  \\
   & H\degree_{comb}  = -4.1MJmol^{-1} \pm 0.2 MJmol^{-1}                                       \\
   & H\degree_{comb}  = -4.1MJ \pm 0.2 MJ
\end{align*}

Table \ref*{tab:experCalc} presents the calculated experimental molar enthalpies of combustion for each trial.

\begin{table}[H]
  \centering
  \begin{tabular}{|l|l|}
    \hline
    Trial Number & \begin{tabular}[c]{@{}l@{}}Experimental molar enthalpy of combustion $/MJ$ \\ $(\pm 0.2 MJ)$ \end{tabular} \\
    \hline
    1            & -4.1                                                                                                       \\
    \hline
    2            & -4.2                                                                                                       \\
    \hline
    3            & -4.8
    \\
    \hline
  \end{tabular}
  \caption{Calculated experimental molar enthalpy of combustion for each trial}
  \label{tab:experCalc}
\end{table}

\par The average experimental molar enthalpy of combustion for Paraffin Wax can then be calculated, as shown below.
\begin{align*}
  \overline{\Delta H\degree}_{comb} & = \frac{\sum \Delta H\degree_{comb}}{N}
  \\
  \text{where }                     & \overline{\Delta H\degree}_{comb} = \text{ calculated average experimental molar enthalpy /MJ}
  \\
                                    & \Delta H\degree_{comb} = \text{ calculated experimental molar enthalpy for a trial /MJ}
  \\
                                    & N = \text{ number of trials conducted}
  \\
  \overline{\Delta H\degree}_{comb} & = \frac{(-4.1MJ \pm 0.2 MJ) + (-4.2 \pm 0.2 MJ) + (-4.8 \pm 0.2 MJ)}{3}
  \\
                                    & = -4.4MJ \pm 0.2 MJ
\end{align*}

\par Note that the decision of what the uncertainty of the averaged value would be is based on all of
the uncertainties of the individual values calculated from each trial, in which it would be reasonable
for the final averaged value to follow the same pattern.
\\
\par Finally, the percent error can be calculated using this averaged value as the experimental molar enthalpy of combustion.

\begin{align*}
  \%_{err} & = \left|\frac{experimental - theoretical}{theoretical} \times 100\%\right|
  \\
           & = \left|\frac{-4.4MJ - (-14.7MJ)}{-14.7MJ} \times 100\%\right|
  \\
           & = 70.3\%
\end{align*}

\section{Conclusion}

\subsection{Summary}

\par
The theoretical value of the molar enthalpy of the combustion of Paraffin Wax
(\ce{C25H52(s)}) was determined to be -14.7MJ.

\par Experimentally, the molar enthalpy of the combustion of Paraffin Wax was determined to
be $-4.4MJ \pm 0.2MJ$.

The experiment value calculated, including any values within the range given by the uncertainty,
is way less (in magnitude) from the theoretical value, yielding a percent error of $70.3\%$. This indicates
that the experiment was unsuccessful.

Analyzing the precision and accuracy of the calculated experimental molar enthalpies of combustion,
it becomes evident that the data was precise but not accurate, as the three values
$(-4.1MJ, -4.2MJ, -4.8MJ)$ did not vary that much from each other given only a
$0.7MJ$ range. Of course, more trials would be necessary to be able to achieve
a better assessment of the precision in the experimental molar enthalpies of combustion;
however, this range from just the 3 trials is enough to indicate that the data was precise
but not accurate, therefore indicating that the unsuccessfulness of the experiment was due
to systematic errors.

Another thing to add that was interesting was that even though the first trial was not timed
accurately, its experimental molar enthalpy of combustion happened to be closer to that of the
second trial than that of the third trial.

\subsection{Evaluation}

\subsubsection{Delay between lighting candle and assembling calorimeter}

One notable systematic source of error was the presence of a delay between lighting the candle and
lowering the ring on the ring stand, which held the can and the aluminium foil and therefore needed
to be lowered in order to form an enclosed calorimeter (at least enclosed to the calorimeter's greatest
potential).

This source of error is one that would cause the experimental value to be lower than the theoretical
value. Because the paraffin wax started combusting before the calorimeter was enclosed, some of the
heat released from the combustion reaction is going into the surroundings of the room rather than the
surroundings of the calorimeter. This entails that some of the heat that should've transferred into
the can of the water was instead released into the air. What this means is that the value of
the final temperature of the water measured will be lower than expected, directly causing the value of
$\Delta T$ to be lower than expected as well for both the can and the water.

Referring back to the equation
$$
  H\degree_{comb} = \frac{m_{water}c_{water}\Delta T_{water} + m_{Al}c_{Al}\Delta T_{Al}}{n}
$$
it becomes evident that the value of $H\degree_{comb}$ will be lower than expected as well, as since
$\Delta T_{water}$ and $\Delta T_{Al}$ ultimately are the same value, then
$$
  H\degree_{comb} = \frac{\Delta T(m_{water}c_{water} + m_{Al}c_{Al})}{n}
$$
therefore indicating a proportional relationship between $H\degree_{comb}$ and $\Delta T$.

\subsubsection{Opening at the top of the calorimeter}

The opening at the top of the calorimeter that is the circular space formed between the can
and the ring is a systemic error that explains a reason why the experimental value was lower than
the theoretical value.

This source of error is similar to the delay between lighting the candle and assembling the
calorimeter, in which some of the heat released from the combustion reaction is going into the
surroundings of the room rather than the surroundings of the calorimeter. However, rather than
a large amount of heat being lost during a small interval of time, the heat lost during
the experiment was due to part of the heat increasing the temperature of the can and the
water and part of the heat escaping through the opening. Nevertheless, the value of
the final temperature of the water measured will be lower than expected, directly causing the value of
$\Delta T$ to be lower than expected as well for both the can and the water.

Referring back to the equation
$$
  H\degree_{comb} = \frac{m_{water}c_{water}\Delta T_{water} + m_{Al}c_{Al}\Delta T_{Al}}{n}
$$
it becomes evident that the value of $H\degree_{comb}$ will be lower than expected as well, as since
$\Delta T_{water}$ and $\Delta T_{Al}$ ultimately are the same value, then
$$
  H\degree_{comb} = \frac{\Delta T(m_{water}c_{water} + m_{Al}c_{Al})}{n}
$$
therefore indicating a proportional relationship between $H\degree_{comb}$ and $\Delta T$.

\subsection{Suggested Improvements}

\subsubsection{Solution to delay between lighting candle and assembling calorimeter}

The main concern had for lighting the candle was safety in being able to light the candle
within a calorimeter. Our immediate thought was to create a temporary opening in the
aluminium foil and light the candle through the hole; however, the safety concern was that
the experimenter would not be able to completely see where their hand would be within the calorimeter
while attempting to light the candle.

A potential solution for the source of error of there being a delay between lighting the
candle and assembling the calorimeter while maintaining a level of safety is to use a
BBQ lighter instead of a small butane lighter and follow the approach of lighting through
a small gap in the aluminium foil. The small radius of a BBQ lighter would allow for the
experimenter to be able to see within the calorimeter despite it being fully assembled,
and the length of the BBQ lighter allows for the experimenter to keep their hand outside of
the calorimeter.


\subsubsection{Solution to opening at the top of the calorimeter}

A potential solution to address the source of error of the opening at the top of the calorimeter
is to place a stable cover on top of the ring. This gets rid of the opening at the top of the calorimeter
while removing the necessity of the more expensive bomb calorimeter. One consideration is
required for this approach, however; as the thermometer within the can still needs to be visible.
This suggests that there will still need to be an opening at the top of the calorimeter, but
a significantly smaller hole may significantly assist in dedicating more of the combustion reaction's
released heat towards the kinetic energy of the water and can.

%----------------------------------------------------------------------------------------
%	BIBLIOGRAPHY
%----------------------------------------------------------------------------------------

\printbibliography % Output the bibliography

%----------------------------------------------------------------------------------------

\end{document}